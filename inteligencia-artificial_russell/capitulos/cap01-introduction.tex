\section{Introdução à IA}

\subsection{O que é Inteligência Artificial?}
\begin{conceptbox}
\underline{\underline{\textbf{Definição}}} : IA é o estudo de agentes que recebem percepções do ambiente e realizam ações.
\end{conceptbox}

\begin{quotebox}
“Artificial intelligence is the study of agents that receive percepts from the environment and perform actions.”\\
\hfill -- Russell \& Norvig
\end{quotebox}

$\hookrightarrow$

\begin{enumerate}
    \item É \underline{positivo} ($0 \geq $) ou \underline{negativo} ($0 \leq$).
    \item $0$ é \underline{positivo} e \underline{negativo}.
    \item $b \geq a \text{ }(\text{ou } a \leq b) \rightarrow a-b \geq 0$.
    \item $b > a \text{ }(\text{ou }  a < b) \rightarrow b \geq a \text{ e } b \ne a$.
\end{enumerate}

\begin{enumerate}[label=(\roman*)]
        \item Madonna não é cantora.
        \item Acordei tarde e perdi a aula.
        \item Assitia Netflix ou estudava para a prova de Lógica.
        \item Se você fez atividade física, então está suado. / Se você usa casaco, então está frio.
        \item A luz está acesa se, e somente se, o interruptor está ligado.(ChatGPT)
\end{enumerate} 



\subsection{Exemplo de IA no Mundo Real}
\begin{examplebox}
Um agente inteligente em um veículo autônomo analisa sensores para decidir quando frear ou acelerar em um cruzamento.
\end{examplebox}

\section{Agentes Inteligentes}

\subsection{Modelo de Agente}
\begin{conceptbox}
\textbf{Agente:} Entidade que percebe o ambiente via sensores e age sobre ele via atuadores.
\end{conceptbox}

\begin{diagrambox}
%\includegraphics[width=0.7\textwidth]{imagens/agente-diagrama.png}\\
\textit{Figura: Arquitetura simplificada de um agente inteligente}
\end{diagrambox}

\begin{definitionbox}
Uma definição pode ser inserida aqui, utilizando um visual uniforme, limpo e elegante para os conceitos importantes do texto.
\end{definitionbox}

\section{Notas e Observações Gerais}
Espaço reservado para reflexões, pontos a revisar, questões, etc.
