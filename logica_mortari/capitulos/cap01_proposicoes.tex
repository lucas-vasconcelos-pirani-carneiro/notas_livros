\section{Introdução}
Neste capítulo, iremos caracterizar \underline{o que é a lógica} e sua relação com raciocínio, indferência 
e argumento.

\subsection{O que é a lógica ?}
Dificil apresentar uma definição exata

Vamos começar com uma ideia ainda que não seja muito precisa



\subsection{Raciocínio e Inferência}

\subsection{Argumentos}

\subsection{Sentenças, Proposições e Enunciados}


\subsection{O que é uma proposição?}
\begin{defbox}
Uma \textbf{proposição} é toda sentença declarativa que pode ser classificada como verdadeira ou falsa, mas não ambas.
\end{defbox}

\begin{examplebox}
"O sol é uma estrela."
\end{examplebox}

\subsection{Operadores Lógicos}
\begin{defbox}
\textbf{Negação $(\neg)$}: operador que inverte o valor lógico da proposição.
\end{defbox}

\begin{ttbox}
\begin{tabular}{c|c}
$p$ & $\neg p$ \\
\hline
V & F \\
F & V \\
\end{tabular}
\end{ttbox}

\subsection{Teoremas Básicos da Lógica}
\begin{theorembox}
Lei de De Morgan: $\neg(p \land q) \equiv (\neg p) \lor (\neg q)$
\end{theorembox}

\begin{proofbox}
Basta construir a tabela-verdade e comparar os valores das duas proposições.
\end{proofbox}

\section{Notas e Observações Pessoais}
\begin{itemize}[leftmargin=1.5em]
    \item Adicione aqui exemplos práticos, dúvidas, ou reflexões durante o estudo de lógica!
\end{itemize}
