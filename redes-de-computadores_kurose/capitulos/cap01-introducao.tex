\section{Introdução às Redes de Computadores}

\begin{defbox}
Redes de computadores permitem a troca de informações entre dispositivos, utilizando protocolos de comunicação.
\end{defbox}

\begin{examplebox}
Enviar um e-mail ou acessar um site são exemplos cotidianos do uso de redes de computadores.
\end{examplebox}

\section{O Approach Top-Down}
\begin{notebox}
O livro explora as redes a partir da camada de aplicação, descendo para transporte, rede, enlace, e física.
\end{notebox}

\begin{lstlisting}[style=dracula, language=Python]
def saudacao(nome):
    # Função que cumprimenta
    print(f"Olá, {nome}!")
\end{lstlisting}

\section{Camada de Aplicação}

\subsection{HTTP}
\begin{protocolbox}
\textbf{HyperText Transfer Protocol (HTTP):} Protocolo para transferência de páginas web baseado em requisições e respostas entre cliente e servidor. Métodos principais: GET, POST, PUT, DELETE.
\end{protocolbox}

\begin{examplebox}
Quando você acessa um site, o navegador faz uma requisição HTTP GET ao servidor do site, que então responde com o conteúdo da página.
\end{examplebox}

\subsection{DNS}
\begin{protocolbox}
\textbf{Domain Name System (DNS):} Serviço que traduz nomes de domínio legíveis (ex: www.exemplo.com) para endereços IP da rede.
\end{protocolbox}

\section{Camada de Transporte}
\subsection{TCP vs UDP}
\begin{defbox}
\textbf{TCP:} Protocolo orientado à conexão, garante entrega e ordem dos dados.\\
\textbf{UDP:} Protocolo sem conexão, sem garantia de entrega ou ordem, porém mais rápido.
\end{defbox}

\begin{notebox}
TCP é utilizado para aplicações que exigem confiabilidade (ex: web, e-mail). UDP é preferido quando latência mínima é fundamental (ex: streaming, jogos).
\end{notebox}

\section{Diagramas/Protocolos em Redes}
\begin{diagrambox}
%\includegraphics[width=0.7\textwidth]{imagens/exemplo-diagrama-rede.png}\\
\textit{Figura: Exemplo de comunicação via HTTP}
\end{diagrambox}

\section{Notas Pessoais}
Espaço para dúvidas, reflexões, insights de aula/prática, pegadinhas frequentes da disciplina, etc.
